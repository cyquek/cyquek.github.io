\documentclass[12pt,a4paper]{article}
\usepackage{amsmath}
\usepackage{amssymb}
\usepackage[pdftex]{graphicx}
\usepackage{setspace}
\usepackage[marginal,bottom]{footmisc}
\usepackage[para]{threeparttable}
\usepackage{fancyhdr}
\usepackage[a4paper, left= 1.0in, right = 1.0in, bottom = 1.0in, top = 1.0in,includefoot,nomarginpar,includemp]{geometry}
\usepackage{bm}
\usepackage{sectsty}
\usepackage{lscape}
\usepackage{color}
\usepackage[sectionbib,comma]{natbib}
%\usepackage{indentfirst}
\sectionfont{\raggedright}
\setlength{\bibhang}{0.5in} \DeclareMathOperator*{\VAR}{VAR}
\DeclareMathOperator*{\COV}{COV} \DeclareMathOperator*{\corr}{corr}
\DeclareMathOperator*{\s}{s} \DeclareMathOperator*{\se}{se}
\setlength{\parindent}{0.5in} \interfootnotelinepenalty=10000
\makeatletter
\renewcommand*{\l@figure}{\@dottedtocline{1}{0em}{3.8em}}
\makeatother \makeatletter
\renewcommand*{\l@table}{\@dottedtocline{1}{0em}{3.8em}}
\makeatother

\singlespacing

\begin{document}
\thispagestyle{plain}
\begin{center}
\textbf{Department of Economics} \\
\textsc{The University of Melbourne}
\end{center}
\vspace{0.1in}
\begin{center}
\textbf{ECOM30001/ECOM90001: Basic Econometrics} \\
\textbf{Semester 1, 2022}
\end{center}
\vspace{0.1in}
\begin{center}
\textsc{Solutions: Tutorial 3}
\end{center}
\noindent \hrulefill


\noindent This tutorial reviews some fundamental concepts for the basic linear model, including conducting hypothesis tests about a single coefficient, using the econometrics
software package \textsc{R} that we will be using in this
subject. \vspace{0.1in}

\noindent This tutorial requires one data file:
\begin{itemize}
\item[-] \texttt{cocaine.csv}
\end{itemize}
This file can be obtained from the LMS subject page.  \vspace{0.1in}

\noindent In addition the R script file \texttt{tut3.R} provides the program code necessary to complete the tutorial. This R script file uses the following package(s) which need to be installed prior to running the R script file:
\begin{align*}
\mbox{\texttt{stargazer}} & : \mbox{for easily generating summary statistics for an R data file} \\
\end{align*}
These can be installed directly in \textsc{RStudio} from the packages tab or by using the command \texttt{install.packages()} and inserting the name of the package in the brackets.

 \newpage

\subsubsection*{Question 1 (Hill, Griffiths \& Judge Exercise 5.5)}
\renewcommand{\labelenumi}{\alph{enumi})}
\renewcommand{\labelenumii}{\roman{enumii})}

Consider the following econometric model for prices of owner-occupied homes in a metropolitan area surrounding a major city.
\begin{align*}
\mbox{ VALUE}_{i} & = \beta_{0} + \beta_{1}\,\mbox{ CRIME}_{i} +
\beta_{2}\,\mbox{
NITOX}_{i} \\
 &+ \beta_{3}\,\mbox{ ROOMS}_{i} + \beta_{4}\,\mbox{ AGE}_{i} + \beta_{5}\,\mbox{ DIST}_{i}
 \\
 & + \beta_{6}\,\mbox{ ACCESS}_{i} + \beta_{7}\,\mbox{ TAX}_{i}
+ \beta_{8}\,\mbox{ PTRATIO}_{i} + \varepsilon_{i}
\end{align*}

These variables are defined as:
\begin{align*}
\mbox{ VALUE } & = \mbox{ median value of owner-occupied homes in \$'000's }\\
\mbox{ CRIME } & = \mbox{ per-capita crime rate }\\
\mbox{ NITOX } & = \mbox{ nitric oxides concentration (parts per million }\\
\mbox{ ROOMS } & = \mbox{ average number of rooms per dwelling }\\
\mbox{ AGE } & = \mbox{ proportion of owner-occupied dwelliingsbuilt prior to 1940 }\\
\mbox{ DIST } & = \mbox{ weighted distance to five employment centres }\\
\mbox{ ACCESS }& = \mbox{ index of accessability to radial highways } \\
\mbox{ TAX } & = \mbox{ full-value property tax rate per \$10,000 }\\
\mbox{ PTRATIO } & = \mbox{ pupil-teacher ratio by town  }\\
\end{align*}

\noindent Suppose you have data on 506 local census areas within the major city. An OLS regression provides the following estimates:
\begin{center}
\begin{tabular}{c c c c c}\\
\multicolumn{5}{l}{Dependent Variable: VALUE}\\
\multicolumn{5}{l}{Sample: 1 506}\\
\multicolumn{5}{l}{Included Observations : 506}\\ \hline

Variable & Coefficient & Std.Error & t-statistic & Prob. \\ \hline

C & 28.40666 & 5.365948 & 5.293875 & 0.0000 \\

CRIME & -0.183449 & 0.036489 & -5.027548 & 0.0000 \\

NITOX & -22.81088 & 4.160741 & -5.482407 & 0.0000 \\

ROOMS & 6.371512 & 0.392387 & 16.23784 & 0.0000 \\

AGE & -0.047750 & 0.014102 & -3.386085 & 0.0008 \\

DIST & -1.335269 & 0.200147 & -6.671448 & 0.0000 \\

ACCESS & 0.272282 & 0.072276 & 3.376250 & 0.0002 \\

TAX & -0.012592 & 0.003770 & -3.339939 & 0.0009 \\

PTRATIO & -1.176787 & 0.139415 & -8.440868 & 0.0000 \\ \hline
\end{tabular}
\end{center}
\begin{enumerate}
\item Report briefly how each of the explanatory variables affects
the value of a home. \vspace{0.1in}

\noindent \textbf{Solution} Recall that the dependent variable
(median value of owner-occupied homes) is measured in \$'000's.
\begin{itemize}
\item $b_{1} = -0.184$---as the per-capita crime rate increases by one unit, the median
home value falls by \$183.45.

\item $b_{2} = -22.81088$---a one unit increase in the nitric oxide concentration leads
to a decline in median value of \$22,810.

\item $b_{3} = 6.371512$---increasing the average number of rooms by one unit raises the
median value by \$6,372

\item $b_{4} = -0.04775$---a one unit increase in the proportion of owner-occupied dwellings built prior to 1940 reduces the median home value by \$47.75

\item $b_{5} = -1.335269$---for every unit of weighted distance from five employment centers, the median home value declines by \$1335.27

\item $b_{6} = 0.272282$---a one unit increase in the index of accessability to radial
highways raises median value by \$272.82

\item $b_{7} = -0.012592$---a higher property tax rate per \$10,000 lowers the median home value

\item $b_{8} = -1.176787$---a one unit increase in the pupil-teacher ratio lowers median  value
by \$1176.78
\end{itemize}

\item Test the hypothesis that increasing the average number of rooms by one
changes the median value of a house by $\$7,500$. Your answer should
clearly state the null and alternative hypotheses, the distribution
of the test statistic, and your decision. \vspace{0.1in}

\noindent \textbf{Solution}: This hypothesis may be represented as
$H_{0}: \beta_{3} = 7.5$ since the dependent variable is measured in
\$1000. Consider a two-sided alternative:
\begin{enumerate}
\item specify $H_{0}$ and $H_{A}$:
\begin{flalign*}
& H_{0}: \beta_{3} = 7.5 & H_{A}: \beta_{3} \neq 7.5
\end{flalign*}
\item the test statistic :
\[
t = \frac{b_{3} - 7.5}{\se(b_{3})} \thicksim t(N-9)
\]
\item the level of significance:
\begin{flalign*}
& \alpha = 0.05 & t_{c} \thickapprox 1.96
\end{flalign*}
with degrees of freedom $N-9 = 497$. Reject $H_{0}$ if $t \geq 1.96$ or $t \leq -1.96$. Note that the file \texttt{tut3.R} provides an exact critical value of $t_{c} = 1.964749$ and $-t_{c} =-1.964749$.

\item regression results
\begin{flalign*}
& b_{3} =6.371512 & \se(b_{3}) = 0.392387
\end{flalign*}
\item calculate the sample value of the test statistic
\[
t = \frac{b_{3} -7.5}{\se(b_{3})}  = \frac{-1.128488}{0.392387}
=-2.87596
\]
\item apply the decision rule:
\[
t<-t_{c}
\] so \textbf{reject the null hypothesis}.
The data are not consistent with the hypothesis that changing the
number of rooms raises the median value of a house by \$7,500. \vspace{0.1in}

\noindent Alternatively, the p-value fore this test is given by $p=0.0042$. Since $p<0.05$, then reject $H_{0}$. Check the file
\texttt{tut3.R} for the calculation of this p-value.
\end{enumerate}

\item Test the hypothesis that reducing the
pupil-teacher ratio by 10 will increase the median value of a house by more
than $\$10,000$. Your answer should clearly state the null and
alternative hypotheses, the distribution of the test statistic, and
your decision. \vspace{0.1in}

\noindent \textbf{Solution}: We want to test $H_{A}: \beta_{8} <
-1$:
\begin{enumerate}
\item specify $H_{0}$ and $H_{A}$:
\begin{flalign*}
& H_{0}: \beta_{8}\geq -1 & H_{A}: \beta_{8} < -1
\end{flalign*}
\item the test statistic :
\[
t = \frac{b_{8} -(-1)}{\se(b_{8})} \thicksim t(N-9)
\]
\item the level of significance:
\begin{flalign*}
& \alpha = 0.05 & t_{c} \thickapprox 1.645
\end{flalign*}
with degrees of freedom $N-9 = 497$. Reject $H_{0}$ if $t \leq
-1.645$ since this is a one-tailed test. Note that the file \texttt{tut3.R} provides an exact critical value of $t_{c} = -1.647925$.

 \item regression results
\begin{flalign*}
& b_{8} =-1.176787 & \se(b_{3}) = 0.139415
\end{flalign*}
\item calculate the sample value of the test statistic
\[
t = \frac{b_{8} + 1}{\se(b_{8})}  = \frac{ -0.17679}{0.139415}
=-1.2681
\]
\item apply the decision rule. We will reject $H_{0}$ when $t<-t_{c}$. Since $t >
-t_{c}$, \textbf{do not reject the null hypothesis}. The data are not
consistent with the hypothesis that reducing the pupil-teacher
ratio by 10 will increase the median home value by more than \$10,000. \vspace{0.1in}

\noindent Alternatively, the p-value fore this test is given by $p=0.1027$. Since $p>0.05$, then do not reject $H_{0}$. Check the file \texttt{tut3.R} for the calculation of this p-value.
\end{enumerate}
\end{enumerate}

\noindent \textbf{Note}: The hypothesis tests for parts (c) and (d)
should be conducted at the 5\% level of significance.
\newpage
\subsubsection*{Question 2}


Illicit drugs are increasingly being sold on crypto-markets on the `dark-web'. These markets facilitate anonymous buying and selling and are characterised by the following characteristics:
\begin{itemize}
\item [-] anonymous internet browsing
\item [-] payments in virtual crypto-currencies (such as Bitcoin)
\item [-] payments to third-party vendors (intermediaries)
\item [-] vendor feedback systems (ratings)
\end{itemize}

\noindent This question examines the market for cocaine using a single platform on the dark-web, during July 2017. Consider the following econometric model:
\[
\mbox{price}_{i} = \beta_{0} + \beta_{1}\,\mbox{quant}_{i} + \beta_{2}\,\mbox{qual}_{i} + \beta_{3}\,\mbox{rating}_{i} + \varepsilon_{i}
\]
where:
\begin{align*}
\mbox{price } & = \mbox{price per gram in \$USD for a cocaine sale}\\
\mbox{quant} & = \mbox{number of grams of coccaine in a given sale}\\
\mbox{qual} & = \mbox{quality of the coccaine expressed as a percentage purity}\\
\mbox{rating} & = \mbox{rating of the seller on a five-hundred point scale, $0$ to $500$}
\end{align*}
The data required to complete this question are located on the
subject page. The data file is called \texttt{cocaine.csv}. \vspace{0.1in}

\noindent The hypothesis tests for parts (d), (e), and (f) should be conducted at the 5\% level of significance.
\begin{enumerate}
\item What signs do you expect for $\beta_{1}$, $\beta_{2}$ and
$\beta_{3}$? \vspace{0.1in}

\noindent \textbf{Solution}:We expect the following signs for the
population coefficients:
\begin{itemize}
\item $\beta_{1} < 0$---as the number of grams in a given sale increases by one unit, the
expected price per gram will fall (for a given quality and seller rating). This suggests that
there is a quantity discount for large sales.

\item $\beta_{2}  > 0$---as the quality index increases by one percentage point, the expected price will
increase (for a given quantity and seller rating). We expect that, all else equal,
relatively more pure cocaine should command a higher unit price.

\item $\beta_{3} > 0$---as the seller rating increases by one unit, the expected price will increase (for a given
quantity and quality). This suggests that the reputations of the sellers are important for buyers such that, all else equal,  they are
willing to pay a premium to purchase from more reliable sellers.
\end{itemize}
\item Using \textsc{R} estimate the econometric model. Report
and interpret the coefficient estimates. Do the signs of your
estimated coefficients turn out as you expect? \vspace {0.1in}

\noindent \textbf{Solution}: The estimated regression results are
reported in Figure \ref{Fi:quest2}.
\begin{figure}
\begin{center}
\setlength{\unitlength}{1mm}
\includegraphics[scale = 0.75]{tut3_reg1}
\end{center}
\caption{Question 2: OLS Regression Results} \label{Fi:quest2}
\end{figure}
\begin{itemize}
\item $b_{1}  = -0.0798$---as the number of grams in a given sale increases by one
unit, the price per gram will fall by \$USD 0.0798. This marginal effect is relatively small, which might reflect the
reduced risk faced by sellers on the `dark web'.

\item $b_{2}  = 0.7636$---as the quality index increases by one percentage point, the price per gram
will increase by \$USD 0.7636.

\item $b_{3} = 0.2179$---as seller rating increases by one unit (on a 500-point scale), the price per gram increases by
\$USD 0.2179.
\end{itemize}

\noindent The estimated coefficients have the signs that we would expect.

\item What proportion of the variation in cocaine price is explained
by variation in quantity, quality, and seller rating? \vspace{0.1in}

\noindent \textbf{Solution}: Using the $R^{2} = 0.1307$ from the
regression results in Figure \ref{Fi:quest2}, the proportion of the
variation in the cocaine unit price that is explained by variation
in the quantity, quality, and seller rating is 13.09\%.

\item It is claimed that the greater the number of sales, the higher
the risk of getting caught; and thus, sellers are willing to accept
a lower price if they can make sales in greater quantities. Test
this hypothesis. Your answer should clearly state the null and
alternative hypotheses, the distribution of the test statistic, and
your decision. \vspace{0.1in}
 \newpage
\noindent \textbf{Solution}: This hypothesis may be represented as
a one-sided test:
\begin{enumerate}
\item specify $H_{0}$ and $H_{A}$:
\begin{flalign*}
& H_{0}: \beta_{1} \geq 0 & H_{A}: \beta_{1} < 0
\end{flalign*}
\item the test statistic :
\[
t = \frac{b_{1} - 0}{\se(b_{1})} \thicksim t(N-4)
\]
\item the level of significance:
\begin{flalign*}
& \alpha = 0.05 & t_{c} \thickapprox 1.6449
\end{flalign*}
with degrees of freedom $(N-4) = 1,405$. Reject $H_{0}$ if $t \leq
-1.6449$. Note that the file \texttt{tut3.R} provides an exact critical value of $t_{c} = -1.6459$.

\item regression results
\begin{flalign*}
& b_{1} =-0.0798 & \se(b_{1}) = 0.0071
\end{flalign*}
\item calculate the sample value of the test statistic
\[
t = \frac{b_{1} -0}{\se(b_{1})}  = -11.2844
\]
\item apply the decision rule:
\[
t < -t_{c} \] \textbf{reject the null hypothesis}. The data are
consistent with the hypothesis that sellers are willing to accept a
lower price if they can make sales in larger quantities. \vspace{0.1in}

\noindent Alternatively, the p-value fore this test is given by $p=0.0000$. Since $p<0.05$, then reject $H_{0}$. Check the file \texttt{tut3.R} for the calculation of this p-value.
\end{enumerate}

\item Test the hypothesis that a premium is paid
for better quality cocaine. Your answer should clearly state the
null and alternative hypotheses, the distribution of the test
statistic, and your decision. \vspace{0.1in}

\noindent \textbf{Solution}: This hypothesis may be represented as
a one-sided test.
\begin{enumerate}
\item specify $H_{0}$ and $H_{A}$:
\begin{flalign*}
& H_{0}: \beta_{2} \leq 0 & H_{A}: \beta_{2} > 0
\end{flalign*}
\item the test statistic :
\[
t = \frac{b_{2} - 0}{\se(b_{2})} \thicksim t(N-4)
\]
\item the level of significance:
\begin{flalign*}
& \alpha = 0.05 & t_{c} \thickapprox 1.6449
\end{flalign*}
with degrees of freedom $(N-4) = 1,405$. Reject $H_{0}$ if $t \geq
1.6449$. Note that the file \texttt{tut3.R} provides an exact critical value of $t_{c} = 1.6459$.

\item regression results
\begin{flalign*}
& b_{2} =0.7636 & \se(b_{2}) = 0.0965
\end{flalign*}
\item calculate the sample value of the test statistic
\[
t = \frac{b_{2} -0}{\se(b_{2})}  = 7.9143
\]
\item apply the decision rule:
\[
t > t_{c}
\] \textbf{reject the null hypothesis}. The data are consistent with the hypothesis that a premium is paid for
better quality cocaine. \vspace{0.1in}

\noindent Alternatively, the p-value fore this test is given by $p=0.00000$. Since $p<0.05$, then reject $H_{0}$. Check the file \texttt{tut3.R} for the calculation of this p-value.
\end{enumerate}

\item Test the hypothesis that, controlling for quality and quantity, seller rating is an important
determinant of price. Your answer should clearly state the
null and alternative hypotheses, the distribution of the test
statistic, and your decision.  \vspace{0.1in}

\noindent \textbf{Solution}: This hypothesis may be represented as
a one-sided test.
\begin{enumerate}
\item specify $H_{0}$ and $H_{A}$:
\begin{flalign*}
& H_{0}: \beta_{3}=  0 & H_{A}: \beta_{3} \ne = 0
\end{flalign*}
\item the test statistic :
\[
t = \frac{b_{3} - 0}{\se(b_{3})} \thicksim t(N-4)
\]
\item the level of significance:
\begin{flalign*}
& \alpha = 0.05 & t_{c} \thickapprox 1.96
\end{flalign*}
with degrees of freedom $(N-4) = 1,405$. Reject $H_{0}$ if $t \geq 1.96$ or $t \leq -1.96$. Note that the file \texttt{tut3.R} provides an exact critical value of $t_{c} = 1.9617$.

\item regression results
\begin{flalign*}
& b_{3} =0.2179 & \se(b_{3}) = 0.0475
\end{flalign*}
\item calculate the sample value of the test statistic
\[
t = \frac{b_{3} -0}{\se(b_{3})}  = 4.5848
\]
\item apply the decision rule:
\[
t > t_{c}
\] \textbf{reject the null hypothesis}. The data are consistent with the hypothesis that seller rating is an important determinant of price.\vspace{0.1in}

\noindent Alternatively, the p-value fore this test is given by $p=0.00000$. Since $p<0.05$, then reject $H_{0}$. Check the file \texttt{tut3.R} for the calculation of this p-value.
\end{enumerate}
\end{enumerate}



\end{document}
