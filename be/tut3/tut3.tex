\documentclass[12pt,a4paper]{article}
\usepackage{amsmath}
\usepackage{amssymb}
\usepackage[pdftex]{graphicx}
\usepackage{setspace}
\usepackage[marginal,bottom]{footmisc}
\usepackage[para]{threeparttable}
\usepackage{fancyhdr}
\usepackage[a4paper, left= 1.0in, right = 1.0in, bottom = 1.0in, top = 1.0in,includefoot,nomarginpar,includemp]{geometry}
\usepackage{bm}
\usepackage{sectsty}
\usepackage{lscape}
\usepackage{color}
\usepackage[sectionbib,comma]{natbib}
%\usepackage{indentfirst}
\sectionfont{\raggedright}
\setlength{\bibhang}{0.5in} \DeclareMathOperator*{\VAR}{VAR}
\DeclareMathOperator*{\COV}{COV} \DeclareMathOperator*{\corr}{corr}
\DeclareMathOperator*{\s}{s} \DeclareMathOperator*{\se}{se}
\setlength{\parindent}{0.5in} \interfootnotelinepenalty=10000
\makeatletter
\renewcommand*{\l@figure}{\@dottedtocline{1}{0em}{3.8em}}
\makeatother \makeatletter
\renewcommand*{\l@table}{\@dottedtocline{1}{0em}{3.8em}}
\makeatother

\singlespacing

\begin{document}
\thispagestyle{plain}
\begin{center}
\textbf{Department of Economics} \\
\textsc{The University of Melbourne}
\end{center}
\vspace{0.1in}
\begin{center}
\textbf{ECOM30001/ECOM90001: Basic Econometrics} \\
\textbf{Semester 1, 2022}
\end{center}
\vspace{0.1in}
\begin{center}
\textsc{Tutorial 3}
\end{center}
\noindent \hrulefill

\noindent This tutorial reviews some fundamental concepts for the basic linear model, including conducting hypothesis tests about a single coefficient, using the econometrics
software package \textsc{R} that we will be using in this
subject. \vspace{0.1in}

\noindent This tutorial requires one data file:
\begin{itemize}
\item[-] \texttt{cocaine.csv}
\end{itemize}
This file can be obtained from the Canvas subject page.  \vspace{0.1in}

\noindent In addition the R script file \texttt{tut3.R} provides the program code necessary to complete the tutorial. This R script file uses the following package(s) which need to be installed prior to running the R script file:
\begin{align*}
\mbox{\texttt{stargazer}} & : \mbox{for easily generating summary statistics for an R data file} \\
\end{align*}
This package can be installed directly in \textsc{RStudio} from the packages tab or by using the command \texttt{install.packages()} and inserting the name of the package in the brackets.

 \newpage
\subsubsection*{Question 1 (Hill, Griffiths \& Lim Exercise 5.5)}
\renewcommand{\labelenumi}{\alph{enumi})}

Consider the following econometric model for prices of owner-occupied homes in a metropolitan area surrounding a major city.
\begin{align*}
\mbox{ VALUE}_{i} & = \beta_{0} + \beta_{1}\,\mbox{ CRIME}_{i} +
\beta_{2}\,\mbox{
NITOX}_{i} \\
 &+ \beta_{3}\,\mbox{ ROOMS}_{i} + \beta_{4}\,\mbox{ AGE}_{i} + \beta_{5}\,\mbox{ DIST}_{i}
 \\
 & + \beta_{6}\,\mbox{ ACCESS}_{i} + \beta_{7}\,\mbox{ TAX}_{i}
+ \beta_{8}\,\mbox{ PTRATIO}_{i} + \varepsilon_{i}
\end{align*}

These variables are defined as:
\begin{align*}
\mbox{ VALUE } & = \mbox{ median value of owner-occupied homes in \$'000's }\\
\mbox{ CRIME } & = \mbox{ per-capita crime rate }\\
\mbox{ NITOX } & = \mbox{ nitric oxides concentration (parts per million) }\\
\mbox{ ROOMS } & = \mbox{ average number of rooms per dwelling }\\
\mbox{ AGE } & = \mbox{ proportion of owner-occupied dwellings built prior to 1940 }\\
\mbox{ DIST } & = \mbox{ weighted distance to five employment centres }\\
\mbox{ ACCESS }& = \mbox{ index of accessability to radial highways } \\
\mbox{ TAX } & = \mbox{ full-value property tax rate per \$10,000 }\\
\mbox{ PTRATIO } & = \mbox{ pupil-teacher ratio by town  }\\
\end{align*}

\noindent Suppose you have data on 506 local census areas within the major city. An OLS regression provides the following estimates:
\begin{center}
\begin{tabular}{c c c c c}\\
\multicolumn{5}{l}{Dependent Variable: VALUE}\\
\multicolumn{5}{l}{Sample: 1 506}\\
\multicolumn{5}{l}{Included Observations : 506}\\ \hline

Variable & Coefficient & Std.Error & t-statistic & Prob. \\ \hline

C & 28.40666 & 5.365948 & 5.293875 & 0.0000 \\

CRIME & -0.183449 & 0.036489 & -5.027548 & 0.0000 \\

NITOX & -22.81088 & 4.160741 & -5.482407 & 0.0000 \\

ROOMS & 6.371512 & 0.392387 & 16.23784 & 0.0000 \\

AGE & -0.047750 & 0.014102 & -3.386085 & 0.0008 \\

DIST & -1.335269 & 0.200147 & -6.671448 & 0.0000 \\

ACCESS & 0.272282 & 0.072276 & 3.376250 & 0.0002 \\

TAX & -0.012592 & 0.003770 & -3.339939 & 0.0009 \\

PTRATIO & -1.176787 & 0.139415 & -8.440868 & 0.0000 \\ \hline
\end{tabular}
\end{center}
\begin{enumerate}
\item Report briefly how each of the explanatory variables affects
the value of a home. \vspace{0.1in}

\item Test the hypothesis that increasing the average number of rooms by one
changes the median value of a house by $\$7,500$. Your answer should
clearly state the null and alternative hypotheses, the distribution
of the test statistic, and your decision.

\item Test the hypothesis that reducing the
pupil-teacher ratio by 10 will increase the median value of a house by more
than $\$10,000$. Your answer should clearly state the null and
alternative hypotheses, the distribution of the test statistic, and
your decision.
\end{enumerate}

\noindent \textbf{Note}: The hypothesis tests for parts (b) and (c)
should be conducted at the 5\% level of significance.

\subsubsection*{Question 2}

Illicit drugs are increasingly being sold on crypto-markets on the `dark-web'. These markets facilitate anonymous buying and selling and are characterised by the following characteristics:
\begin{itemize}
\item [-] anonymous internet browsing
\item [-] payments in virtual crypto-currencies (such as Bitcoin)
\item [-] payments to third-party vendors (intermediaries)
\item [-] vendor feedback systems (ratings)
\end{itemize}

\noindent This question examines the market for cocaine using a single platform on the dark-web, during July 2017. Consider the following econometric model:
\[
\mbox{price}_{i} = \beta_{0} + \beta_{1}\,\mbox{quant}_{i} + \beta_{2}\,\mbox{qual}_{i} + \beta_{3}\,\mbox{rating}_{i} + \varepsilon_{i}
\]
where:
\begin{align*}
\mbox{price } & = \mbox{price per gram in \$USD for a cocaine sale}\\
\mbox{quant} & = \mbox{number of grams of coccaine in a given sale}\\
\mbox{qual} & = \mbox{quality of the coccaine expressed as a percentage purity}\\
\mbox{rating} & = \mbox{rating of the seller on a five-hundred point scale, $0$ to $500$}
\end{align*}
The data required to complete this question are located on the
subject page. The data file is called \texttt{cocaine.csv}. \vspace{0.1in}

\noindent The hypothesis tests for parts (d), (e), and  (f) should be conducted at the 5\% level of significance.
\begin{enumerate}
\item What signs do you expect for $\beta_{1}$, $\beta_{2}$ and
$\beta_{3}$?
\item Using \textsc{R}, estimate the econometric model. Report
and interpret the coefficient estimates. Do the signs of your
estimated coefficients turn out as you expect?

\item What proportion of the variation in cocaine price is explained
by variation in quantity, quality, and seller rating?

\item It is claimed that the greater the number of sales, the higher
the risk of getting caught; and thus, sellers are willing to accept
a lower price if they can make sales in greater quantities. Test
this hypothesis. Your answer should clearly state the null and
alternative hypotheses, the distribution of the test statistic, and
your decision.

\item Test the hypothesis that a premium is paid
for better quality cocaine. Your answer should clearly state the
null and alternative hypotheses, the distribution of the test
statistic, and your decision.

\item Test the hypothesis that, controlling for quality and quantity, seller rating is an important
determinant of price. Your answer should clearly state the
null and alternative hypotheses, the distribution of the test
statistic, and your decision.
\end{enumerate}


\end{document}
